% File name: Tube Challenge/Report/Report.tex
% Report detailing the project etc
% Author: adh
% Date: Tue 02 Jul 2013 16:29

\documentclass[a4paper,11pt]{article}  % Standard document class
\usepackage[english]{babel}            % Set document language
\usepackage{fullpage}                  % Set up page for small margins etc

\usepackage{graphicx}                  % For including images in document
\usepackage{placeins}                  % Allows use of \FloatBarrier
% to avoid images or tables
% moving into next section
\usepackage{subfig}                    % For subfigures...

\usepackage{amsmath}                   % For improving maths/formula typesetting
\usepackage{tabularx}                  % Table changing package

\usepackage{algpseudocode}             % For producing algorithms/flowcharts
\usepackage{listings}                  % For including source code in document

\usepackage{todonotes}

% Provide command for scientific notation
\providecommand{\e}[1]{\ensuremath{\times10^{#1}}}
\providecommand{\degrees}{\ensuremath{^{\circ}}}

% Define title here:
\title{Tube Challenge 2013}
\author{James Griffith \and Andy Holt \and Chris Murkin \and Dominic
  Newman \and Jack Robinson}
\date{June - August 2013}

\begin{document}

% generate title
\maketitle

\section*{The Challenge}

Since he was a young boy, James Griffith has dreamed of visiting every
Tube station in London in a single day. Now, at the age of 21, he's
hoping to make this dream a reality. But he doesn't just want to visit
every station in a day, he's planing to break the world record by
visiting every station in the fastest time, using only public
transport.

The London Underground is the world's oldest underground
railway. Opening in 1863, and now comprising $402\,\mathrm{km}$ of
track, the Tube forms a vast network of underground tunnels and
carries over a billion passengers per year through London. 2013 marks
the 150$^{\mathrm{th}}$ year of operation of the Underground, and is a
fitting time to attempt the challenge.

The challenge is simple: to visit all 270 stations on the Tube network
in a single day. To beat the record, he needs to do it in
under \{\ldots\}\todo{find out current record time}.

But being simple by no means makes the challenge easy. With such a
complex transport network and so many passengers, delays are
inevitable and unpredictable. A delay of just a few minutes in the
chosen route could render the whole challenge unachievable due to
missed trains and connections. The physical element of the challenge
is also gruelling: equivalent to running a marathon through busy
London stations, up and down thousands of steps and sprinting to make
fast connections. Another significant issue is shear boredom: over 16
hours of sitting on Tubes in a day will almost certainly make James
want to quit; taking a good book will be of utmost importance.

\section*{The Team}

With such a huge task ahead of him, James has enlisted the support of
some friends. Each brings a unique skill, vital for the completion of
the challenge.

\section*{The Strategy}

\end{document}